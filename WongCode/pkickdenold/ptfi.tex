%\Documentclass[preprint,showpacs,aps]{revtex4}
%\documentclass[twocolumn,showpacs,aps]{revtex4}
\documentclass[showpacs,preprintnumbers,amsmath,amssymb,floatfix]{revtex4}
%\documentclass[preprint,showpacs,preprintnumbers,amsmath,amssymb,floatfix]{revtex4}
%\documentclass[showpacs,preprintnumbers,amsmath,amssymb]{revtex4}

\headheight=2.0cm

\usepackage{color}   %option added by cyw eg \pagecolor{blue} \color{red}
\usepackage{graphicx}% Include figure files
\usepackage{dcolumn}% Align table columns on decimal point
\usepackage{bm}% bold math

\begin{document}
\def\bbox#1{\hbox{\boldmath${#1}$}}
\def\blambda{{\hbox{\boldmath $\lambda$}}}
\def\eeta{{\hbox{\boldmath $\eta$}}}
\def\bxi{{\hbox{\boldmath $\xi$}}}
\def\bzeta{{\hbox{\boldmath $\zeta$}}}

\title{ 
Further studies of the Momentum Kick Model
}


\author{Cheuk-Yin Wong}
\date{\today}

\begin{abstract}

We apply the momentum kick model to the region of low transverse
momenta to separate the near-side jet associated particles into a jet
and a ridge component.  The experimnetal data of the transverse
distribution up to $p_t$ of 4 GeV and the azimuthal distribution in
the mid-rapidity region can be well described by the momneutm kick
model.  The momentum kick model further predicts the continuation of
the ridge structure up to forward rapditities.

\end{abstract}

\pacs{ 25.75.Gz 25.75.Dw }


\maketitle

\section{Introduction}

In order to get some estimates on the forward rapidity ridge
structure, we need to rely on associated particles with low $p_t$, for
which the number of counts can be large enough for meaningful
measurements.  The difficulty with high $p_t$ and large $\eta$ arises
because the counting rate for large $p_t$ and large rapidity is not
high enough to minimize statistical errors.  

The region of low $p_t$ is also interesting on theoretical grounds as
any extension of the momentum kick model to other kinematic regions of
azimuthal angles will need a good understanding of the interplay
between different contributions in the low $p_t$ region.

We therefore need a reasonable description of the associated particles
with low transverse momenta.  One wishes to ask whether the momentum
kick model can be applied to the region of low $p_t$, and what kind of
corrections are expected when we apply the momentum kick model to the
region of low $p_t$.


\section{The Momentum Kick Model}


\begin{enumerate}

\item
A near-side jet occurs near the medium surface and the jet collides
with partons in the medium on its way to the detector.

\item
The jet-parton collision samples the momentum distribution of the
collided medium partons at the moment of the jet-parton collision.
Because of the condition for the occurrence of the near-side jet,
these collided medium partons are near the surface and each collided
parton suffers at most one collision with the jet.

\item
The jet-parton collision imparts a momentum kick ${\bf q}$ to the
collided parton in the direction of the jet.  The momentum kick
modifies the parton initial momentum distribution $P_i(\bbox{p}_i)$ to
turn it into the collided parton final momentum distribution
$P_f(\bbox{p}_f)$.  After the collided partons hadronize and escape
from the surface without further collisions, they materialize as ridge
particles which retain the collided parton final momentum
distribution.

\end{enumerate}

We shall describe the initial momentum distribution of partons on the
right-hand side of Eq.\ (\ref{pf}) by a Gaussian distribution in
rapidity $y_i$ with a width parameter (the standard deviation)
$\sigma_y$, a thermal transverse momentum distribution characterized
by a `temperature' $T$, and a uniform azimuthal distribution in
$\phi_i$.  
The final momentum distribution is
therefore
\begin{eqnarray}
\label{final}
\frac{dN_f}{d\eta_f d\phi_f p_{tf}dp_{tf}} 
=\left [ A_i e^{-y_i^2/2\sigma_y^2} \frac{ e^{-m_{ti}/T}}{m_{ti}}
\frac{E_f}{E_i} \right ]_{\bbox{p}_i=\bbox{p}_f-q \bbox{e}_{jet}}
\sqrt{1-\frac{m^2}{m_{tf}^2 \cosh^2 y_f}},
\end{eqnarray}
where
\begin{eqnarray}
A_i &=&\frac{N_i e^{m/T}} {(2\pi)^{3/2}\sigma_y T} ,
\end{eqnarray}
where $N_i$ is a total number of partons that collide with the jet.
The initial momentum $\bbox{p}_i=(p_{i1},p_{i2},p_{i3})$ is
related to the final momentum $\bbox{p}_f=(p_{f1},p_{f2},p_{f3})$ and
the trigger jet rapidity $\eta_{jet}$ by
\begin{subequations}
\begin{eqnarray}
p_{i1}&=&p_{f1}-\frac{q}{\cosh \eta_{jet}},\\
p_{i2}&=&p_{f2},\\
p_{i3}&=&p_{f3}-\frac{q\sinh \eta_{jet}}{\cosh \eta_{jet}},
\end{eqnarray}
\end{subequations}
We can obtain $dN/d\Delta \eta d\Delta \phi p_t dp_t$ in terms of
$\Delta \eta=\eta - \eta_{jet}$ and $\Delta \phi = \phi-\phi_{jet}$,
relative to the trigger jet by a simple change of variables.
 The relevant physical quantities are then the magnitude of the
 momentum kick $q$ along the jet direction imparted by the jet to the
 collided parton, and the initial parton momentum distribution
 represented by the rapidity width parameter $\sigma_y$ and the
 transverse momentum temperature $T$.

\section{Considerations for the Region of Small $p_t$}

We consider a jet-parton collsiion in which the jet imparts a total
momentum kick ${\bf q}_{tot}$ onto a parton of initial momentum ${\bf
p}_i$.  Then parton subsequently appears as a hadron in the detector
after the parton udnergoes a parton-hadron transition, which for
convenince can be generally call a parton-hadron duality transition.

In the 'hadronization' process of parton-hadron duality transition,
the parton is expected to lose ${\bf q}_h$ amount of momentum.  The
final parton (hadron) momentum as it emerges from the emdium, ${\bf p}_f$, is
related to the initial parton momentum ${\bf p}_i$ by
\begin{eqnarray}
{\bf p}_f= {\bf p}_i+{\bf q}_{tot}-{\bf q}_h.
\end{eqnarray}
We expect the total momentum kick ${\bf q}_{tot}$ to be in the form of
a cone along the jet direction and can be approximately represented as
along the jet direction with a magnitude $q_{tot}$,
\begin{eqnarray}
{\bf q}_{tot}\sim q_{tot} {\bf e}_{jet}
\end{eqnarray}
 We expect the hadronization is likely to drag the parton, in a
direction opposite the direction of parton momeion.  It is reasonable
to assume that on the average the momentum loss to parton-hadron
duality transition will also be likely to be opposite to the jet
direction and 
\begin{eqnarray}
{\bf q}_h \sim q_h {\bf e}_jet.  
\end{eqnarray}
As a consquence, the
final parton momentum as it emerges from the emdium is ${\bf p}_f$
related to the initial parton momentum ${\bf p}_i$ by
\begin{eqnarray}
{\bf p}_f)&=& {\bf p}_i+{\bf q}_{tot}-{\bf q}_h\nonumber\\
          &=& {\bf p}_i+ ({q}_{tot}-{q}_h) {\bf e}_{jet}\nonumber\\
          &=& {\bf p}_i+ q {\bf e}_{jet}.
\end{eqnarray}
Thus, the effective magnitude of momentum kick $q$ comprises of the
difference the total momentum kick minus the momentum needed for the
parton-hadron duality transition,
\begin{eqnarray}
q= q_{tot}-q_h. 
\end{eqnarray}
Thus, as the momentum loss due to the the effects of the momentum drag
in the parton-hadron duality transition has be included, the momentum
kick model can be use also up to regions of low transverse momentum,
for simple estimate of the yields in  that region.



\section{Transverse Momentum Distribution}

In the near-side jet region which arises from a jet emerging near the
surface, the partons arises essntially as renmants from the jet
trigger and as partons that has collided one time wiht the jet.  For
brevity, we can abrrviatingly called the first the jet and the latter
the ridge components.  The distribution of the transverse momentum
distribution of these two components have differeent behaviours.  the
similarity of the transverse momentum distributions of $pp$ and
peripheerical collisions allows us to make the simplifying assumption
that the jet component associated with the near-side jet in a
nucleus-nucleus collision should be nearly the same as those in $pp$
collisions.  While total yields from the jet and the ridge compoents
can be separated for high-$p_t$ particles, such a separation is not
yet carried out for low-$p_t$ particles.  We shall attempt to see how
they may be separated.

One expects that as the magnitude of the final tarnsverse parton
momentum is greater than the initial momentum by the amount $q$, the
ridge yield of final partons, $dN/p_t dp_t$, should have a peak around
$p_t\sim q$, which corresponds to $p_t\sim 1 $ GeV.  The data of the
total transverse yield indicates that the initial parton distribution
of the form given by $\exp\{ -\sqrt(m^2+p_t^2)/T \}/\sqrt{m^2+p_t^2}$
will lead to a ridge peak that is much higher than the observed
transverse yield. We find it necessary to modfiy the the initial
parton transverse momentum distirbution at low $p_t$ that preserves
the good shape of the distribution at large $p_t$.  The simplest
modification that can lead to agreement with experimnetal data can be achived by using an initial  transverse momentum distribution of the form
\begin{eqnarray}
\frac { dN_i} {p_t dp_t d\phi}= N_i A
\frac{\exp\{-\sqrt{m^2+p_t^2}/T\}} {\sqrt{m_h^2+p_t^2}}
\end{eqnarray}
where $A$ is the normalization consiatnt defined by
\begin{eqnarray}
\int d \phi p_t dp_t \frac { dN_i} {p_t dp_t d\phi}= N_i. 
\end{eqnarray}


\begin{figure} [h]
\includegraphics[angle=0,scale=0.50]{pkdendndfitheo}
\vspace*{0.0cm}
\caption{ Caption 
}
\end{figure}


To determine the values of $N_i$, $T$ and $m_h$ for our case of
$AU$-$Au$ collision, we use the data of $pp$ collisons to fix the jet
yeidl component. 
The jet ($pp$) 
compnent can be parameterized as
\begin{eqnarray}
\frac{N_i({\rm jet})}{s\eta_{\rm jet}}
=N_i(jet) \frac{1}{\sqrt{2\pi}sigma_{\rm jet}}.
\end{eqnarray}
Speciofically, for $pp$ collisions with associated particles within
$|\eta_{associated}|<1$ and $0.15 < p_t < 4$ GeV, the jet distribution
parameters are
\begin{eqnarray}
N_i(jet)=1.2, ~~~ \sigma_{jet}=0.55 {\rm ~ GeV},
\end{eqnarray}
and for  $2 < p_t < 4$ GeV, the jet component can be parameterized with
\begin{eqnarray}
N_i(jet)=0.7122, ~~~ \sigma_{jet}=0.35 {\rm ~GeV},
\end{eqnarray}
They are shown as the dashed curves in Fig.\ 1(a) and (b).

For our case with $\sigma_y$ taken to be 5.5, we find that $T=0.50$
GeV, and $m_b=0.77$ GeV gives a good description of the transverse
data and the azumthal data, as shown in Fig.\ xx.


\begin{figure} [h]
\includegraphics[angle=0,scale=0.50]{pkdendnptdpt}
\vspace*{0.0cm}
\caption{ Caption 
}
\end{figure}


\begin{figure} [h]
\includegraphics[angle=0,scale=0.50]{dndfiPRLfig1}
\vspace*{0.0cm}
\caption{ Caption 
}
\end{figure}


\begin{figure} [h]
\includegraphics[angle=0,scale=0.50]{dndfiwfq2739}
\vspace*{0.0cm}
\caption{ Caption 
}
\end{figure}



\vspace*{-0.3cm}
\begin{thebibliography}{99}

\bibitem{Ada05}
J. Adams $et~al.$ for the STAR Collaboration, 
Phys. Rev. Lett. {\bf 95}, 152301 (2005). 


\bibitem{Wanf07} 
F. Wang for the STAR Collaboration, Invited talk at the XIth
International Workshop on Correlation and Fluctuation in Multiparticle
Production, Hangzhou, China, November 2007, arXiv:0707.0815.


\end{thebibliography} 


\end{document}
